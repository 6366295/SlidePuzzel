\documentclass[a4paper]{article}
\begin{document}

\section{Opbouw App}
Wanneer de app gestart is, wordt er een menu gedisplayed.
De opties in de main menu zijn: play, settings, highscores, how to play en exit.
    \subsection{Settings}
In de settings menu kan de moeilijkheidsgraad, game formaat en game modus worden aangepast.
Voor de moeilijkheidsgraad zijn er drie niveaus geimplementeerd:
\\
\textbullet De easy moeilijkheidsgraad laat hints toe.
\\
\textbullet De normal moeilijkheidsgraad laat geen hints toe.
\\
\textbullet De hard moeilijkheidsgraad draait het beeld om de tien seconden 90 graden.
\\\\
Het formaat van de game kan aangepast worden naar een drie bij drie schuifpuzzel of een vier bij vier schuifpuzzel.
\\\\
De twee gamemodus die we hebben geimplementeerd is een live image mode en een gallery image mode.
Bij de live image mode wordt de camera beeld van de mobiel ingeladen als de schuifpuzzel.
Bij de gallery image mode kan er een image van het geheugen van de mobiel worden ingeladen.
\\\\
De settings worden opgeslagen via een SQLite database dat wordt aangemaakt.
Hierdoor worden de settings van de app behouden, na het afsluiten van de app.
Later wordt er dieper ingegaan op de database.
    \subsection{Highscores}
De highscores gedeelte van de app is verdeeld onder de eerder genoemde moeilijkheidsgraad.
En daaronder zijn ze weer verdeeld onder drie bij drie of vier bij vier.
Er is een duidelijke onderscheiding van highscores in moeilijkheidsgraad en game formaat.
Hierdoor kan de gebruiker zien hoe snel hij of zij erover gedaan heeft in een specifieke moeilijkheidsgraad en game formaat.
\\\\
De highscores worden ingelezen door een query te sturen naar de game database.
De naam en de bijbehorende tijd wordt dan weergegeven, met een maximum van 5 entries.
De highscores gaan van de snelste tijd naar langzaamste.
    \subsection{Play}
De play knop brengt de gebruiker naar de game zelf.
Hieronder worden de features van de game beschreven en uitgelegd.
        \subsubsection{Game}
De schuifpuzzel zelf gebruikt informatie van de settings om het speelveld te creeeren.
Het beeld wordt verdeelt in tiles.
Aantal tiles hangt af van of het een drie bij drie of vier bij vier game formaat is.
Er is altijd een tile die leeg is.
\\
Er wordt altijd gecontrolleerd of een beweging wel mogelijk is, omdat de enige mogelijkheid naar de lege tile hoort te zijn.
Ook wordt er gecontrolleerd of de puzzel opgelost is.
        \subsubsection{Tile Shuffeler}
De schuifpuzzel zou niet moeilijk zijn als de tiles niet geshuffeld waren.
De implementatie voor de shuffeler werkt als volgd:
\\
Er wordt uitgegaan van het eind resultaat.
Vanuit het eindresultaat wordt er random geschoven.
Of het random geschuif een opgeloste puzzel opleverd wordt altijd gecontrolleerd.
Er is gekozen voor deze manier omdat het random tiles op andere plaats leggen mogelijk een onoplosbare puzzel oplevert.
Maar volgens een voorbeeld app met OpenCV kan die manier ook.
Als er gecontrolleerd wordt of het geen onoplosbare puzzel opleverd.
        \subsubsection{Hints}
In easy mode is er een hint systeem ingebouwd. Dit wordt gedaan met een combinatie van brute force en A* algoritmes.
Hier wordt later dieper op ingegaan
        \subsubsection{Rotatie}
In hard mode draait de live camera beeld of de gallery image iedere tien seconden 90 graden.
Door de draaiing van het beeld, moet de gebruiker zich opnieuw in het plaatje orienteren.
Dit maakt de schuifpuzzel aanzienlijk moeilijker.
        \subsubsection{Freeze camera}
In de live video mode van de schuifpuzzel is er een freeze button geimplementeerd.
Hiermee zet je het beeld van de camera still.
Hierdoor is er geen verschil meer tussen de live camera mode en gallery image mode.
Maar voor de gebruiker is het mischien prettig om het camera beeld af en toe stil te zetten.
        \subsubsection{Geluid}
Er zijn een aantal geluidseffecten geimplementeerd in de game.
Elke keer als de gebruiker een tile in beweging zet, wordt er een schuifgeluid afgespeeld.
In de hard mode wordt er bij elke rotatie een geluidsfragment afgespeeld.
En als de gebruiker de puzzel heeft opgelost, wordt er ook een geluidsfragment afgespeeld.
        \subsubsection{Win scherm}
Wanneer de gebruiker de puzzel heeft opgelost, wordt er een win scherm weergegeven.
De tijd waarover de gebruiker gedaan heeft om de puzzel wordt weergeven.
Op dit scherm kan dan gekozen worden of het spel opnieuw gespeeld wordt met de zelfde settings.
of dat er teruggegaan naar de main menu.
\\\\
De gebruiker kan ook een naam opgeven.
Deze komt samen met de tijd in de highscore tabel van de database.


\end{document}
