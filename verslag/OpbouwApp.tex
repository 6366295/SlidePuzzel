\section{Opbouw applicatie}
Wanneer de applicatie gestart is, wordt er een menu gedisplayed.
De opties in het main menu zijn: play, settings, highscores, how to play en exit.
    \subsection{Settings}
In het settings menu kan de moeilijkheidsgraad, game formaat en game modus worden aangepast.
Voor de moeilijkheidsgraad zijn er drie niveaus geimplementeerd:
	\begin{itemize}
		\item{De easy moeilijkheidsgraad laat hints toe}
		\item{De normal moeilijkheidsgraad laat geen hints toe}
		\marginpar{N.B. hints zijn ook niet toegestaan in hard}
		\item{De hard moeilijkheidsgraad draait het beeld om de tien seconden 90 graden}
	\end{itemize}
Het formaat van de game kan aangepast worden naar een drie bij drie of een vier bij vier schuifpuzzelveld.
Hierbij kan gekozen worden uit twee verschillende gamemodes. De live image mode en de gallery image mode.
De live image mode maakt gebruik van het inkomende camerabeeld. De puzzel zal live meeveranderen tijdens het spelen.
Bij de gallery image mode kan een foto uit het geheugen van de Android device worden ingeladen als puzzel.
De settings worden opgeslagen in een database (\ref{sec:database}). Dit zorgt ervoor dat de settings die door de gebruiker zijn ingesteld worden onthouden, zelfs als de applicatie wordt afgesloten.
    \subsection{Highscores}\label{sec:high}
Highscores worden bepaald door de tijd die nodig is om een puzzel op te lossen. De highscores van de applicatie zijn opgedeeld in verschillende categorie\"en. Ten eerste wordt dit bepaald door de moeilijkheidsgraad. Per moeilijkheidsgraad is vervolgens weer een onderverdeling tussen het drie bij drie en het vier bij vier speelveld. Per categorie worden maximaal de vijf beste tijden getoond, met de snelste tijd bovenaan. Hierdoor kan de gebruiker zien wat de beste tijden zijn in een specifieke moeilijkheidsgraad en game formaat.
    \subsection{Play}
De play knop brengt de gebruiker naar de game zelf.
Hieronder worden de features van de game beschreven en uitgelegd.
        \subsubsection{Game}
De schuifpuzzel zelf gebruikt informatie van de settings om het speelveld te cre\" eeren.
Het beeld wordt verdeeld in tiles. Het aantal tiles wordt bepaald door het gekozen spelformaat. Uit het originele beeld wordt de tile die zich rechtsonder bevindt zwart gemaakt. Vervolgens wordt het beeld gehusseld.
        \subsubsection{Tile Shuffler}
Als de tiles in een vier bij vier veld random worden neergelegd is het mogelijk dat daar een niet oplosbare puzzel uit voortkomt. Daarom is besloten om zowel voor de drie bij drie als de vier bij vier dezelfde manier van husselen te gebruiken, die het onmogelijk maakt om een onoplosbare puzzel te cre\"eeren. De puzzel wordt vanuit zijn beginstaat, waar alle tiles op de goede locatie liggen, willekeurig verschoven, mits het mogelijk is om te schuiven. Nadat er gehusseld is wordt gecontroleerd of er niet toevallig het eindresultaat uit is voortgekomen, is dit wel het geval dan wordt overnieuw gehusseld. OpenCV \cite{opencvpuzzle} kent een slidepuzzel van vier bij vier tiles die gebruik maakt van een ander algoritme om de puzzel te husselen \cite{opencvsolve}. Hierbij kunnen de tiles willekeurig op het veld worden geplaatst waarna gekeken wordt of de puzzel op te lossen valt.
        \subsubsection{Hints}
In easy mode is er een hint systeem ingebouwd (\ref{sec:hint}). Bij het gebruik van deze knop in wordt de best mogelijke zet getoond op dat moment. Herhaling zal uiteindelijk de puzzel oplossen.
        \subsubsection{Rotatie}
Als er is gekozen om te spelen in hard, dan wordt het originele beeld iedere tien seconden 90 graden gedraaid. Dit zorgt ervoor dat de gebruiker zich opnieuw moet ori\"enteren. Dit geeft een compleet nieuwe draai aan het schuifpuzzel concept en maakt de schuifpuzzel aanzienlijk moeilijker.
        \subsubsection{Freeze camera}
In de live image mode van de schuifpuzzel is er de mogelijkheid om de freeze button te gebruiken. Hiermee kan het beeld worden vastgezet zodat als het ware met zelfgemaakte foto gespeeld wordt. Het gebruik van deze knop is bedacht om het spelersgemak te vergroten. Immers is het mogelijk dat het beeld continu veel heen en weer beweegt, bijvoorbeeld tijdens het reizen. Dit zorgt er echter voor dat er geen onderscheidt meer is tussen de gallery image en live image mode. Mede hierdoor is er gekozen om in de highscores geen onderscheid te maken tussen live image en gallery image puzzels.
        \subsubsection{Geluid}
Als tijdens het spelen het geluid aan staat op de Android device, zijn er enkele geluiden te horen. Elke keer als de gebruiker een tile in beweging zet, wordt er een schuifgeluid afgespeeld. Daarnaast wordt er in de hard mode bij elke rotatie een geluidsfragment afgespeeld. Zodra de puzzel is opgelost wordt er een geluidsfragment afgespeeld om aan te geven dat het spel voorbij is.
        \subsubsection{Win scherm}
Zodra de puzzel is opgelost wordt er een scherm weergegeven die de gebruiker laat zien hoe lang er over is gedaan om de puzzel op te lossen. De gebruiker kan zijn naam invullen. Mocht de tijd \'e\'en van de snelste tijden zijn dan wordt deze tijd met bijbehorende ingevoerde naam getoond in het highscores menu (\ref{sec:high}). Vervolgens kan er gekozen worden terug te keren naar het main menu door middel van de knop 'back', of om een nieuw spel te starten door middel van de knop 'new game'. In beide gevallen wordt de tijd met bijbehorende naam goed opgeslagen. Als er wordt gekozen om een nieuw spel te starten, dan begint direct een nieuw spel met dezelfde settings als het afgelopen spel.
