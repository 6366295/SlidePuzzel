\section{Conclusie}
	\subsection{Toevoegingen aan het originele idee}
	Al met al bevat de applicatie alle onderdelen die in het originele idee naar voren waren gekomen. Daarnaast zijn er ook nog enkele toevoegingen gemaakt die de applicatie nog meer verbeteren. De grootste wijziging is het toevoegen en bijhouden van highscores door middel van de tijd die nodig is om een puzzel op te lossen. Daarnaast is de freeze functie tijdens het spelen ook een handige toevoeging voor de gebruiker. Tegelijkertijd met het toevoegen van de freeze functie kwam het idee om ook de mogelijkheid te bieden om met foto's uit het geheugen van de Android device te spelen. Hiermee krijgt de gebruiker ook de mogelijkheid om de slidepuzzel op de klassieke manier te spelen.
	\subsection{Verbeteringen}
	Er zijn nog enkele verbeteringen mogelijk bij deze applicatie. Zo groeit de database momenteel net zo lang door totdat er geen geheugen meer vrij is. Er bevindt zich ook een fout in de gallery image mode. Als er voor deze mode is gekozen en de android device wordt in rechtopstaande positie gehouden tijdens het kiezen van een plaatje uit het geheugen, dan komt het gekozen plaatje na het kiezen niet in beeld. Er wordt dan een camerabeeld zichtbaar dat niet is opgedeeld in tiles en niet werkt als een puzzel. Ditzelfde gebeurd als de Android device niet is ingesteld op automatisch draaien.
	\subsection{Conclusie}
	De applicatie voldoet aan de verwachtingen. Er kan op een eenvoudige en leuke manier met live of met een vast beeld gespeeld worden. In vergelijking met openCV kan deze applicatie beduidend meer. De vergelijking met openCV is in sommige opzichten echter niet terecht omdat er bij openCV slechts een eenvoudig voorbeeld van een vier bij vier slidepuzzel wordt gegeven.
