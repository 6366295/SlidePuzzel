\section{Vergelijking met openCV}
	OpenCV biedt vrij te downloaden applicaties aan voor onder andere Android. Zo ook een applicatie die met een live videobeeld een slidepuzzel maakt. Er zijn echter vele verschillen tussen de applicatie van openCV en onze applicatie.
	\subsection{Size}
	OpenCV levert een 15-puzzle (vier bij vier schuifpuzzel). Ze staan duis alleen een vier bij vier veld toe, waar het bij onze applicatie mogelijk is om ook een drie bij drie veld te kiezen. De tiles die openCV gebruikt zijn daarnaast ook geen vierkantjes maar rechthoeken, bij onze applicatie zijn dit wel vierkantjes.
	\subsection{Difficulty}
	De 15-puzzle van openCV kent slechts \'e\'en enkele moeilijkheidsgraad. Deze valt te vergelijken met onze normale modus. OpenCV biedt daarnaast de mogelijkheid aan om getallen weer te geven op de tiles die aangeven waar ze terecht moeten komen. Onze versie bevat meerdere moeilijkheidsgraden. De easy mode geeft de mogelijkheid aan de gebruiker om de best mogelijke zet op dat moment aan te geven. De hard mode maakt het de gebruiker moeilijker door iedere tien seconden een rotatie uit te voeren op het beeld.
	\subsection{Modes}
	De live 15-puzzle van openCV biedt geen mogelijkheid om een foto in te laden om mee te spelen. De applicatie is immers puur gemaakt om met live beeld te spelen. Onze applicatie heeft uiteindelijk ook een mgoelijkheid gekregen om met een foto op de Android Device te spelen.
	\subsection{Shuffler}
	Het cre\"eeren van een slidepuzzel gebeurd door het willekeurig plaatsen of verschuiven van tiles. OpenCV plaatst alle tiles op willekeurige locaties in het speelveld. Vervolgens wordt er gechecked of het gemaakte veld op te lossen valt. Hiervoor wordt een wiskundige berekening gebruikt \cite{opencvsolve}. Valt het bord op te lossen dan begint het spel, anders wordt het willekeurig plaatsen van tiles herhaald tot er wel een speelbaar veld uit komt. Onze applicatie maakt gebruik van een andere methode om een speelveld te maken. Er wordt uitgegaan van het originele beeld. Vervolgens wordt willekeurig een richting gekozen om naartoe te schuiven. Is het mogelijk deze richting op te schuiven dan gebeurdt dit. Is dit niet mogelijk dan wordt overnieuw een willekeurige richting gekozen. Dit proces herhaald zich totdat er 100 willekeurige stappen zijn gedaan. Vervolgens wordt bekeken of de puzzel niet bij toeval het originele beeld oplevert. Als dit wel het geval is dan wordt er overnieuw gehusseld.
	\subsection{Highscores}
	De applicatie van openCV houdt niet bij hoelang een gebruiker erover doet om de puzzel op te lossen. Dit resulteert er in dat er geen highscores bij worden gehouden. Onze applicatie houdt wel een tijd en highscores bij van de gebruiker. Dit levert voor de gebruiker een leukere speelervaring op, hij zal immers steeds sneller en beter willen presteren.
