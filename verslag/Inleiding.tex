	\section{Inleiding}\label{sec:inleiding}
	In dit verslag wordt gekeken naar de werking en opbouw van de Slidepuzzel applicatie. Hoe het originele idee van een camerabeeld omzetten tot een slidepuzzel is uitgebreid naar allerlei verschillende modes, moeilijkheidsgraden en afmetingen en hoe deze puzzels op te lossen. Daarnaast wordt vergeleken hoe deze applicatie werkt ten opzichte van de openCV versie van de 15-puzzle (een vier bij vier slidepuzzel) voor Android.
	\subsection{Wat is een Slidepuzzel?}
	Een slidepuzzel, of schuifpuzzel, is in feite een plaatje dat in even grote blokjes, of tiles, is verdeeld. E\'en van deze tiles wordt uit het originele plaatje verwijderd, waardoor er met de tiles geschoven kan worden. De puzzel bestaat uit N bij M tiles, afhankelijk van het originele plaatje en de grootte van ieder blokje. De meest voorkomende slidepuzzels zijn de 8-puzzle en de 15-puzzle. Deze bestaan respectievelijk uit drie bij drie en vier bij vier tiles.
	\subsection{Originele concept}
	Het originele idee voor deze slidepuzzel Android applicatie was het statische plaatje te vervangen door een dynamisch veranderend camerabeeld. Hierbij zouden enkele verschillende moeilijkheidsgraden komen. Een mogelijkheid om met hints te spelen, de normale versie (met camerabeeld) en een moeilijke versie met roterend beeld. Daarnaast zou het alleen mogelijk zijn om te spelen met een veld van drie bij drie of vier bij vier tiles. Dit in verband met de rotaties, het is immers overzichtelijker om een vierkant beeld te roteren dan om een rechthoekig beeld te roteren. Grotere speelvelden werden al uitgesloten, omdat de tiles dan te klein zouden worden voor sommige Android devices. Hierbij zouden voor het speleffect enkele geluidseffecten worden toegevoegd.
