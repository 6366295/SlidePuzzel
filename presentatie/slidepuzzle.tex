\documentclass{beamer}[10]

\usepackage{pgf}
\usepackage{graphicx}
\usepackage{multirow}
\usepackage{beamerthemesplit}
\usepackage{color}
\usepackage{amsmath}
\definecolor{bgcolor}{RGB}{51,153,255}
\setbeamercovered{transparant}
\mode<presentation>

\usetheme[numbers,totalnumber,compress,sidebarshades]{PaloAlto}
\setbeamertemplate{footline}[1]
\usecolortheme[named=bgcolor]{structure}
\usefonttheme[onlymath]{serif}
\setbeamertemplate{footline}[page number]{}
\setbeamercovered{transparant}
\setbeamertemplate{blocks}[rounded][shadow=true]
\title{\color{black}{\textbf{Multimedia \\Slide Puzzle}}}

\begin{document}
	\begin{frame}

		\frametitle{\color{black}{\textbf{Slide Puzzle}}}
		Door: \\
		Tom Peerdeman, Mike Trieu,\\
		Freddy de Greef \& Ren\'e Aparicio Saez \\
	\end{frame}

	\begin{frame}
		\frametitle{\color{black}{\textbf{Inleiding}}}
		\begin{itemize}
			\item Wat is een Slidepuzzel?
			\item Mogelijke Settings/modes
			\item SQLite
			\item A* Algoritme
			\item Live demo
			\item vragen
		\end{itemize}
	\end{frame}

	\begin{frame}
		\frametitle{\color{black}{\textbf{Wat is een Slidepuzzel?}}}
		\begin{itemize}
			\item 'Verschoven' puzzelstukjes
			\item N \textbf{x} M puzzel
			\item E\'en leeg blokje
		\end{itemize}
		\begin{itemize}
			\item Onze app: N \textbf{x} N puzzel
		\end{itemize}
	\end{frame}

	\begin{frame}
		\frametitle{\color{black}{\textbf{Mogelijke Settings/modes}}}
		\begin{itemize}
			\item{Difficulties}
			\begin{itemize}
				\item{Easy}
				\item{Normal}
				\item{Hard}
			\end{itemize}
			\item{Sizes}
			\begin{itemize}
				\item{3\textbf{x}3}
				\item{4\textbf{x}4}
			\end{itemize}
			\item{Modes}
			\begin{itemize}
				\item{Live Image}
				\item{Plaatje}
			\end{itemize}
		\end{itemize}
	\end{frame}
\end{document}
